\documentclass[]{article}

\usepackage{amsmath,amssymb,amstext,mathtools} % Lots of math symbols and environments
\usepackage[margin=0.5in]{geometry}

\newcommand*\eval[3]{\left.#1\right\rvert_{#2}^{#3}}

%opening
\title{End to End EKF Formulation}
\author{Benjamin Skikos, Chunshang Li}

\begin{document}

\maketitle

\begin{abstract}
This is the formulation for the EKF layer. It uses a Error State Kalman Filter. The IMU gets integrated into the nominal states, and the EKF estimates the error states. The error states are used to correct the nominal states.
\end{abstract}

\section{States}

\subsection{Nominal States}

\begin{itemize}
	\item $p$ global position expressed w.r.t. pose at $t_0$, more precisely ${}_{I}{p}_{IB}$ where $I$ is the intertial frame and $B$ is the body/vehicle frame
	\item $v$ global velocity expressed w.r.t. pose at $t_0$, more precisely ${}_{I}{v}_{IB}$
	\item $q$ Orientation of the vehicle in quaternion form (w, x, y, z) w.r.t. pose at $t_0$, more precisely ${q}_{IB}$
	\item $b_a$ Accelerometer bias in body frame
	\item $b_g$ Gyroscope bias in body frame
\end{itemize}

\subsection{Error States}
\begin{itemize}
	\item $\delta p$ global position error
	\item $\delta v$ global velocity error
	\item $\delta \theta$ Orientation error in so(3)
	\item $\delta b_a$ Accelerometer bias error
	\item $\delta b_g$ Gyroscope bias error
\end{itemize}

\section{Process Model}

New IMU measurements are integrated into the nominal state at each time step. The error state process model carries forward the uncertainty of each IMU integration, the errors are estimated at each measurement update. The errors estimate are then injected into the nominal state.

Inputs are assumed constant from time $t_{k}$ to time $t_{k+1}$.

\begin{itemize}
	\item $a_{m_k} = a_k + \eta_{a}$ Acceleration as measured by the accelerometer. Modeled with white noise.
	\item $w_{m_k} = w_k + \eta_{w}$ Angular velocity as measured by the gyroscope. Modeled with white noise.
\end{itemize}

\subsection{Nominal States Discrete Process Model}

IMU measurements are integrated to produce predictions of nominal.

\begin{align}
\Delta t &= t_{k+1} - t_{k} \\
\hat{p}_{k+1|k} &= \hat{p}_{k|k} + \hat{v}_{k|k} \Delta t + \frac{1}{2}(R\{\hat{q}_{k|k}\}(a_{m_k} - \hat{b}_{a_{k|k}}) + g) \Delta t^2 \\
\hat{v}_{k+1|k} &=  \hat{v}_{k|k} + (R\{\hat{q}_{k|k}\}(a_{m_k} - \hat{b}_{a_{k|k}}) + g) \Delta t \\
\hat{q}_{k+1|k} &= \hat{q}_{k|k} \otimes q\{(w_{m_k} - \hat{b}_{w_{k|k}}) \Delta t\} \\
\hat{b}_{a_{k+1|k}} &= \hat{b}_{a_{k|k}} \\
\hat{b}_{w_{k+1|k}} &= \hat{b}_{w_{k|k}}
\end{align}

\subsection{Error States Discrete Process Model}
\begin{align}
\delta \hat{p}_{k+1|k} &= \delta \hat{p}_{k|k} + \delta \hat{v}_{k|k} \Delta t \\
\delta \hat{v}_{k+1|k} &= \delta \hat{v}_{k|k} + (-R\{\hat{q}_k\}[a_{m_k} - \hat{b}_{a_{k|k}}]_{\times} \delta \theta - R\{\hat{q}_{k|k}\} \delta \hat{b}_{a_{k|k}}) \Delta t  + v_i\\
\delta \hat{\theta}_{k+1|k} &= R\{(w_{m_k} - \hat{b}_{w_{k|k}})\Delta t\} \delta \theta - \delta \hat{b}_{w_{k|k}} \Delta t + \theta_{i}\\
\delta \hat{b}_{a_{k+1|k}} &= \delta \hat{b}_{a_{k|k}} + a_i \\
\delta \hat{b}_{w_{k+1|k}} &= \delta \hat{b}_{w_{k|k}} + w_i
\end{align}

where the $v_i$, $\theta_{i}$, $a_i$, and $w_i$ are the random impulses resulted from the integration of noise terms from continuous to discrete. let $x$ be the comlete error state, the covariance for the error state process model is given by:

\begin{equation}Q_k = F_i Q_{i_k} F_i^\intercal\end{equation}
\begin{equation}F_i = \frac{\partial x}{\partial i}\end{equation}
\begin{equation}Q_{i_k} = 
\begin{bmatrix}
\sigma_{a}^2 \Delta t^2 I_3  &                           0 &                         0 & 0 \\
                          0  & \sigma_{w}^2 \Delta t^2 I_3 &                         0 & 0 \\
                          0  &                           0 & \sigma_{b_a} \Delta t I_3 & 0 \\
                          0  &                           0 &                         0 & \sigma_{b_w} \Delta t I_3
\end{bmatrix}
\end{equation}

Note that initializing error states to zero will always produce zero predictions, only the uncertainty is carried forward by the process model.

\section{Measurement Model}

The ESKF estimates accumulated error between each measurement update. The error states are injected as follows:

\begin{equation}\hat{x}_{k+1|k+1} = \hat{x}_{k+1|k} \oplus \hat{\delta x}_{k+1|k+1}\end{equation}

\begin{equation}
\begin{bmatrix}
\hat{p}_{k+1|k+1} \\
\hat{v}_{k+1|k+1} \\
\hat{q}_{k+1|k+1} \\
\hat{b}_{a_{k+1|k+1}} \\
\hat{b}_{w_{k+1|k+1}}
\end{bmatrix} =
\begin{bmatrix}
\hat{p}_{k+1|k} + \delta \hat{p}_{k+1|k+1}\\
\hat{v}_{k+1|k} + \delta \hat{v}_{k+1|k+1}\\
\hat{q}_{k+1|k} \otimes q\{\delta \hat{\theta}_{k+1|k+1}\}\\
\hat{b}_{a_{k+1|k}} + \delta \hat{b}_{a_{k+1|k+1}}\\
\hat{b}_{w_{k+1|k}} + \delta \hat{b}_{w_{k+1|k+1}}
\end{bmatrix}
\end{equation}

The neural network is trained to directly output 6 DoF incremental poses in the $t_k$ body frame. let the neural network output at time k be $\Delta x_k$. Note that ESKF uses global pose and velocity error, and local angle error. Since the network estimates 6 DoF incremental poses, we are interested in using the position error $\hat{p}_{k+1|k+1}$ and angle errors $\delta \hat{\theta}_{k+1|k+1}$ as part of the measurement update. We can measure the error by finding the difference between the IMU prediction and network prediction at time $k$. For the position, we have the following:

\begin{align}
p_{k+1|k+1} &= p_{k+1|k} + \delta p \\
R\{q_{k|k}\}(\Delta p_k) + p_{k|k} &= p_{k+1|k} + \delta p \\
\Delta p_k &= R\{q_{k|k}\}^{\intercal}(p_{k+1|k} + \delta p_{} - p_{k|k})
\end{align}

For rotation we are the following:

\begin{align}
q_{k+1|k+1} &= q_{k+1|k} \otimes q\{\delta \theta\} \\
q_{k|k} \otimes q\{\Delta \theta_k\} &= q_{k+1|k} \otimes q\{\delta \theta\} \\
R_{k|k} R\{\Delta \theta_k\} &= R_{k+1|k} R\{\delta \theta\} \\
R_{k|k} R\{\Delta \theta_k\} &= R_{k|k} R\{(w_{m_k}-b_{w_{k|k}}) \Delta t\} R\{\delta \theta\} \\
R\{\Delta \theta_k\} &= R\{(w_{m_k}-b_{w_{k|k}}) \Delta t\} R\{\delta \theta\} \\
\Delta \theta_k &= \mathrm{log}(R\{(w_{m_k}-b_{w_{k|k}}) \Delta t\} R\{\delta \theta\})
\end{align}

The error state is alway equal to zero before the update. Thus, the measurement equation is as follows:
\begin{equation}
h(\delta x = 0) = 
\begin{bmatrix}
\Delta p_k \\
\Delta \theta_k
\end{bmatrix} =
\begin{bmatrix}
C\{q_{k|k}\}^{\intercal}(p_{k+1|k} - p_{k|k}) \\
(w_{m_k}-b_{w_{k|k}}) \Delta t
\end{bmatrix} + \eta_{meas}
\end{equation}

The jacobian of the measurement equation is as follows, where $J_r$ is the right Jacobian of $SO(3)$:

\begin{equation}
H_k = 
\eval{\frac{\partial}{\partial \delta x}}{\delta x = 0}{} \approx
\begin{bmatrix}
R\{q_{k|k}\}^{\intercal} & 0 \\
                       0 & J_r^{-1}\{(w_{m_k}-b_{w_{k|k}}) \Delta t\}
\end{bmatrix}
\end{equation}

\section{To the next time step}
Before proceeding to the next time step, the error state is reset to zero, and the uncertainty is propagated as follows after the reset:

\begin{equation}P_{k+1|k+1}^{reset} = G P_{k+1|k+1} G^{\intercal}\end{equation} 

\begin{equation}G = 
\begin{bmatrix}
I_6  &    0                                        & 0   \\
  0  & I_3 - [\frac{1}{2} \delta \theta]_{\times}  & 0  \\
  0  &    0                                        & I_6  \\
\end{bmatrix}
\end{equation}

\end{document}
